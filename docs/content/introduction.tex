\chapter{Introduction}
\label{chapter:introduction}

Damast is a comprehensive visual analysis system for the collection, analysis, and export of historical data regarding peaceful coexistence of religious groups.
Damast was developed for the digital humanities project \enquote{Dhimmis \& Muslims --- Analysing Multireligious Spaces in the Medieval Muslim World}.
The project was funded by the VolkswagenFoundation within the scope of the \enquote{Mixed Methods} initiative.
The project was a collaboration between the Institute for Medieval History II of the Goethe University in Frankfurt/Main, Germany, and the Institute for Visualization and Interactive Systems at the University of Stuttgart, and took place there from 2018 to 2021.

The objective of this joint project was to develop a novel visualization approach in order to gain new insights on the multi-religious landscapes of the Middle East under Muslim rule during the Middle Ages (7th to 14th century).
In particular, information on multi-religious communities were researched and made available in a database accessible through interactive visualization as well as through a pilot web-based geo-temporal multi-view system to analyze and compare information from multiple sources.
A publicly explorable version\footnote{\url{https://damast.geschichte.hu-berlin.de/}} of the research is available at Humboldt-Universität zu Berlin.
An export of the data collected in the project can be found in the data repository of the University of Stuttgart (DaRUS)~\cite{darus}.

Damast consists of a Flask server backend, which also communicates with a PostgreSQL database, as well as a web-based frontend.
The frontend consists of multiple pages with various functions, including:
an interactive visual analysis component, a table-based database editing interface, a document-based annotation interface for data entry, and various smaller utilities.
Damast also supports generating textual reports based on subsets of the historical data.

The rest of this document is structured as follows:
The structure and contents of the main PostgreSQL database, as well as the user and report databases, are discussed in \cref{chapter:database};
the structure and functionalities of the server backend are discussed in \cref{chapter:backend}; and
the various front-end facilities are discussed in \cref{chapter:frontend}.
