\chapter{REST API Endpoint Documentation}
\label{appendix:rest-api-endpoints}

{\large\noindent\verb!/rest/annotation-suggestion/<int:as_id>!\hfill{}\tikz\node[inner sep=2pt,rounded corners=2pt,draw]{\small\texttt{DELETE}}; \tikz\node[inner sep=2pt,rounded corners=2pt,draw]{\small\texttt{OPTIONS}};}
\nopagebreak[4]
{\small\begin{verbatim}Delete an annotation suggestion.
\end{verbatim}}
\pagebreak[2]{\large\noindent\verb!/rest/annotation/<int:annotation_id>!\hfill{}\tikz\node[inner sep=2pt,rounded corners=2pt,draw]{\small\texttt{DELETE}}; \tikz\node[inner sep=2pt,rounded corners=2pt,draw]{\small\texttt{GET}}; \tikz\node[inner sep=2pt,rounded corners=2pt,draw]{\small\texttt{HEAD}}; \tikz\node[inner sep=2pt,rounded corners=2pt,draw]{\small\texttt{OPTIONS}}; \tikz\node[inner sep=2pt,rounded corners=2pt,draw]{\small\texttt{PATCH}}; \tikz\node[inner sep=2pt,rounded corners=2pt,draw]{\small\texttt{PUT}};}
\nopagebreak[4]
{\small\begin{verbatim}CRUD endpoint to get document metadata for a document.

[all]       @param annotation_id        ID of annotation


GET         @returns                    application/json

Get the annotation data for annotation `annotation_id`. Example payload:

    {
        "id": 1,
        "document_id": 3,
        "span": "(422,431)",
        "comment": "comment content"
    }


PUT         @returns                    application/json

Create a new annotation. Returns the created annotation tuple's ID. Example payload:

    {
        "document_id": 4,
        "span": "[0, 10]",
        "comment": "foo bar"
    }


PATCH       @returns                    205 Reset Content; empty body

Modify an existing annotation. Example payload:

    {
        "comment": "new comment"
    }


DELETE      @returns                    application/json

Delete an annotation, and its connected instance. This will fail if the
instance is still in use by an evidence tuple. Returns the deleted
annotation's and instance's IDs.
\end{verbatim}}
\pagebreak[2]{\large\noindent\verb!/rest/annotator-evidence-list!\hfill{}\tikz\node[inner sep=2pt,rounded corners=2pt,draw]{\small\texttt{GET}}; \tikz\node[inner sep=2pt,rounded corners=2pt,draw]{\small\texttt{HEAD}}; \tikz\node[inner sep=2pt,rounded corners=2pt,draw]{\small\texttt{OPTIONS}};}
\nopagebreak[4]
{\small\begin{verbatim}This endpoint returns a list of `evidence_id`, `document_id` tuples for all
evidence that was created using the annotator; i.e., all evidence whose
instances are connected to annotations.

Example return value excerpt:

  [[8122, 1], [8125, 1], [8145, 2], ...]

\end{verbatim}}
\pagebreak[2]{\large\noindent\verb!/rest/confidence-values!\hfill{}\tikz\node[inner sep=2pt,rounded corners=2pt,draw]{\small\texttt{GET}}; \tikz\node[inner sep=2pt,rounded corners=2pt,draw]{\small\texttt{HEAD}}; \tikz\node[inner sep=2pt,rounded corners=2pt,draw]{\small\texttt{OPTIONS}};}
\nopagebreak[4]
{\small\begin{verbatim}Get a list of all confidence values.

@returns        application/json

Example return value:

  ['false', 'uncertain', 'contested', 'probable', 'certain']

\end{verbatim}}
\pagebreak[2]{\large\noindent\verb!/rest/document/<int:document_id>!\hfill{}\tikz\node[inner sep=2pt,rounded corners=2pt,draw]{\small\texttt{GET}}; \tikz\node[inner sep=2pt,rounded corners=2pt,draw]{\small\texttt{HEAD}}; \tikz\node[inner sep=2pt,rounded corners=2pt,draw]{\small\texttt{OPTIONS}};}
\nopagebreak[4]
{\small\begin{verbatim}CRUD endpoint to get document content for a document.

GET       @param document_id            ID of document
          @returns                      document.content_type

This endpoint gets the actual document data for the document with ID
`document_id`. Because those documents can be quite large, this endpoint
supports the `Range` HTTP header, but only in the following forms:

    For `text/plain` type documents: `Range: bytes=[n]-[m]`
    For `text/html` type documents: `Range: content-characters=[n]-[m]`

where `[n]` and `[m]` are both optional numbers. A range
header with more than one range is not supported at the moment. The return
code of the request is either 200 or 206 for successful requests, 416, 400
or 404 for unsuccessful requests.

SIDENOTE: The `<unit>=-<suffix-length>` variant of the `Range` header is
currently not supported correctly, but instead is interpreted as
`<unit>=0-<suffix-length>`.
\end{verbatim}}
\pagebreak[2]{\large\noindent\verb!/rest/document/<int:document_id>/annotation-list!\hfill{}\tikz\node[inner sep=2pt,rounded corners=2pt,draw]{\small\texttt{GET}}; \tikz\node[inner sep=2pt,rounded corners=2pt,draw]{\small\texttt{HEAD}}; \tikz\node[inner sep=2pt,rounded corners=2pt,draw]{\small\texttt{OPTIONS}};}
\nopagebreak[4]
{\small\begin{verbatim}REST endpoint to get a list of annotations associated with a document.

This is a slice of the `annotation_overview` VIEW. It contains all the data
from the `annotation` table, alongside the connected instance IDs and
evidence ID. The `span` property is an array with the first and last
inclusive index of the annotation. As an instance can be part of multiple
evidence tuples (i.e., an annotation can be in multiple groups), the
`evidence_ids` is an array.

Example return value for `/rest/document/1/annotation-list`:

  [
    {
      "id": 1,
      "document_id": 1,
      "span": [ 0, 4 ],
      "comment": "test comment",
      "place_instance_id": null,
      "person_instance_id": 1623,
      "religion_instance_id": null,
      "time_group_id": null,
      "evidence_ids": [ 3611 ]
    },
    ...
  ]


This endpoint can also be requested as in LD-JSON format. In this case, it
is assumed to be required for the Recogito tool. NOTE that to properly show
the annotations in Recogito, the `{ mode: 'pre' }` configuration value must
be set, as the DOM annotator we use internally respects all white-space in
the document without collapsing it.

Example return value excerpt for

  GET /rest/document/7/annotation-list
  Accept: application/ld+json

  [
    {
      "@context": "http://www.w3.org/ns/anno.jsonld",
      "body": [
        {
          "purpose": "tagging",
          "type": "TextualBody",
          "value": "Religion"
        },
        {
          "purpose": "tagging",
          "type": "TextualBody",
          "value": "Christianity"
        }
      ],
      "id": "#31847",
      "target": {
        "selector": [
          {
            "end": 479,
            "start": 474,
            "type": "TextPositionSelector"
          },
          {
            "exact": "Amen.",
            "type": "TextQuoteSelector"
          }
        ]
      },
      "type": "Annotation"
    },
    ...
    {
      "@context": "http://www.w3.org/ns/anno.jsonld",
      "body": [
        {
          "purpose": "tagging",
          "type": "TextualBody",
          "value": "religion"
        }
      ],
      "id": "#5843_religion",
      "motivation": "linking",
      "target": [
        {
          "id": "#31826"
        },
        {
          "id": "#31840"
        }
      ],
      "type": "Annotation"
    },
    ...
  ]
\end{verbatim}}
\pagebreak[2]{\large\noindent\verb!/rest/document/<int:document_id>/annotation-suggestion-list!\hfill{}\tikz\node[inner sep=2pt,rounded corners=2pt,draw]{\small\texttt{GET}}; \tikz\node[inner sep=2pt,rounded corners=2pt,draw]{\small\texttt{HEAD}}; \tikz\node[inner sep=2pt,rounded corners=2pt,draw]{\small\texttt{OPTIONS}};}
\nopagebreak[4]
{\small\begin{verbatim}REST endpoint to get a list of annotation suggestions associated with a document.


@returns application/json


Exemplary return value except:

  [
     {
       "document_id": 3,
       "entity_id": 772,
       "id": 830681,
       "source": [
         "name"
       ],
       "span": [
         1372645,
         1372657
       ],
       "type": "place"
     },
     ...
  ]
\end{verbatim}}
\pagebreak[2]{\large\noindent\verb!/rest/document/<int:document_id>/evidence-list!\hfill{}\tikz\node[inner sep=2pt,rounded corners=2pt,draw]{\small\texttt{GET}}; \tikz\node[inner sep=2pt,rounded corners=2pt,draw]{\small\texttt{HEAD}}; \tikz\node[inner sep=2pt,rounded corners=2pt,draw]{\small\texttt{OPTIONS}};}
\nopagebreak[4]
{\small\begin{verbatim}Get a list of evidence tuples based on a document.

@param              document_id     ID of document
@returns            application/json

An evidence tuple is based on a document if any of its instances'
annotations is located in that document.

Example return value excerpt:

  [
    {
        "comment": "Bischof nachgewiesen",
        "id": 1632,
        "interpretation_confidence": null,
        "person_instance_id": null,
        "place_instance_id": 1632,
        "religion_instance_id": 1632,
        "time_group_id": 1632,
        "visible": true
    },
    ...
  ]
\end{verbatim}}
\pagebreak[2]{\large\noindent\verb!/rest/document/<int:document_id>/metadata!\hfill{}\tikz\node[inner sep=2pt,rounded corners=2pt,draw]{\small\texttt{GET}}; \tikz\node[inner sep=2pt,rounded corners=2pt,draw]{\small\texttt{HEAD}}; \tikz\node[inner sep=2pt,rounded corners=2pt,draw]{\small\texttt{OPTIONS}};}
\nopagebreak[4]
{\small\begin{verbatim}CRUD endpoint to get document metadata for a document.


GET       @param document_id            ID of document
          @returns                      application/json


Example return value for `/document/3/metadata`:

  {
    "comment": "Michael Rabo, Chronography",
    "content_length": 2926379,
    "content_type": "text/html;charset=UTF-8",
    "default_source_confidence": null,
    "document_version": 1,
    "id": 3,
    "source_id": 67,
    "source_name": "Michael der Syrer; Moosa, Matti (2014): The Syriac ...",
    "source_type": "Primary source"
  }
\end{verbatim}}
\pagebreak[2]{\large\noindent\verb!/rest/document/list!\hfill{}\tikz\node[inner sep=2pt,rounded corners=2pt,draw]{\small\texttt{GET}}; \tikz\node[inner sep=2pt,rounded corners=2pt,draw]{\small\texttt{HEAD}}; \tikz\node[inner sep=2pt,rounded corners=2pt,draw]{\small\texttt{OPTIONS}};}
\nopagebreak[4]
{\small\begin{verbatim}GET a list of all documents.


Example return value excerpt:

    [
      {
        "comment": "Michael Rabo, Chronography",
        "content_length": 3012062,
        "content_type": "text/html;charset=UTF-8",
        "default_source_confidence": null,
        "document_version": 1,
        "id": 3,
        "source_id": 67,
        "source_name": "Michael der Syrer; Moosa, Matti (2014): The Syriac ...",
        "source_type": "Primary source"
      },
      ...
    ]
\end{verbatim}}
\pagebreak[2]{\large\noindent\verb!/rest/dump/!\hfill{}\tikz\node[inner sep=2pt,rounded corners=2pt,draw]{\small\texttt{GET}}; \tikz\node[inner sep=2pt,rounded corners=2pt,draw]{\small\texttt{HEAD}}; \tikz\node[inner sep=2pt,rounded corners=2pt,draw]{\small\texttt{OPTIONS}};}
\nopagebreak[4]
{\small\begin{verbatim}Get a database dump of the PostgreSQL database.

@returns application/sql

This dumps the database and returns the SQL. If the user requesting it is
an administrator, the entire database is dumped, including the user and
provenance tables.
\end{verbatim}}
\pagebreak[2]{\large\noindent\verb!/rest/evidence-list!\hfill{}\tikz\node[inner sep=2pt,rounded corners=2pt,draw]{\small\texttt{GET}}; \tikz\node[inner sep=2pt,rounded corners=2pt,draw]{\small\texttt{HEAD}}; \tikz\node[inner sep=2pt,rounded corners=2pt,draw]{\small\texttt{OPTIONS}};}
\nopagebreak[4]
{\small\begin{verbatim}Get a list of compact evidence tuples from the view `place_religion_overview`.

@returns            application/json

This replaces the `/PlaceReligion` API endpoint of the old servlet
implementation. Returns a JSON array of objects with a place ID, evidence
tuple ID, religion ID, and a time span. Only evidences with
`evidence.visible`, `place.visible` and `place_type.visible` are listed!

Example return value excerpt:

  [
    {
      "place_id": 1,
      "religion_id": 4,
      "time_span": {
        "end": 1200,
        "start": 800
      },
      "tuple_id": 1,
      "source_ids": [12]
    },
    ...
  ]
\end{verbatim}}
\pagebreak[2]{\large\noindent\verb!/rest/evidence/<int:evidence_id>!\hfill{}\tikz\node[inner sep=2pt,rounded corners=2pt,draw]{\small\texttt{DELETE}}; \tikz\node[inner sep=2pt,rounded corners=2pt,draw]{\small\texttt{GET}}; \tikz\node[inner sep=2pt,rounded corners=2pt,draw]{\small\texttt{HEAD}}; \tikz\node[inner sep=2pt,rounded corners=2pt,draw]{\small\texttt{OPTIONS}}; \tikz\node[inner sep=2pt,rounded corners=2pt,draw]{\small\texttt{PATCH}}; \tikz\node[inner sep=2pt,rounded corners=2pt,draw]{\small\texttt{PUT}};}
\nopagebreak[4]
{\small\begin{verbatim}CRUD endpoint to manipulate evidence tuples.

[all]     @param evidence_id      ID of evidence tuple, 0 or `None` for PUT

C/PUT     @payload                application/json
          @returns                application/json

Create a new evidence tuple. `place_instance_id` and `religion_instance_id`
are required fields, the rest (`person_instance_id`, `time_group_id`,
`comment`, `interpretation_confidence`, `visible`) is optional. Returns the
IDs for the created evidence.

Exemplary payload for `PUT /evidence`:

  {
    "place_instance_id": 202,
    "religion_instance_id": 7,
    "person_instance_id": 12,
    "time_group_id": 4,
    "interpretation_confidence": "probable",
    "visible": true,
    "comment": "evidence comment: test"
  }


R/GET     @returns                application/json

Get one evidence tuple, specified by `evidence_id`. Request takes no
payload and returns a JSON object with data from a bunch of tables
(`place`, `religion`, `time_span`, `source_instance`, `source`, and
intermediary tables).

Exemplary reply for `GET /evidence/64`:

  {
      "evidence_id": 64,
      "interpretation_confidence": null,
      "location_confidence": null,
      "place_attribution_confidence": null,
      "place_comment": "",
      "place_geoloc": null,
      "place_id": 46,
      "place_instance_comment": null,
      "place_name": "Beth Sayda",
      "religion_confidence": null,
      "religion_id": 4,
      "religion_instance_comment": null,
      "religion_name": "Syriac Orthodox Church",
      "sources": [
          {
              "source_id": 1,
              "source_instance_comment": "",
              "source_name": "OCN = Fiey, [...]",
              "source_page": "179"
          },
          {
              "source_id": 2,
              "source_instance_comment": "Levenq, G. (1935). : s. v. Bêth Saida. [...],
              "source_name": "DHGE = Aubert, [...],
              "source_page": "8, 1239-1240"
          }
      ],
      "time_group_id": 64,
      "time_spans": [
          {
              "comment": null,
              "confidence": null,
              "end": 1277,
              "start": 1261
          }
      ]
  }


U/PATCH   @payload                application/json
          @returns                application/json

Update one or more of the fields `comment`, `interpretation_confidence`,
`visible`, `person_instance_id`, or `time_group_id`. The
`religion_instance_id` and `place_instance_id` CANNOT be updated for an
existing evidence tuple.

Exemplary payload for `PATCH /evidence/12345`:

  {
    "visible": false,
    "comment": "updated comment...",
    "person_instance_id": 1234
  }


D/DELETE  @param                  cascade=0|1
          @returns                application/json

Delete evidence. If the `cascade` parameter is 1, also delete all related
entities. Write `user_action` log, return a JSON with all deleted IDs.

\end{verbatim}}
\pagebreak[2]{\large\noindent\verb!/rest/evidence/<int:evidence_id>/source-instances!\hfill{}\tikz\node[inner sep=2pt,rounded corners=2pt,draw]{\small\texttt{GET}}; \tikz\node[inner sep=2pt,rounded corners=2pt,draw]{\small\texttt{HEAD}}; \tikz\node[inner sep=2pt,rounded corners=2pt,draw]{\small\texttt{OPTIONS}};}
\nopagebreak[4]
{\small\begin{verbatim}Get all source instance entries for evidence `evidence_id`.

@param evidence_id            ID of evidence tuple
@returns                      application/json

Exemplary return value for `GET /evidence/3662/source-instances`:

  [
    {
      "comment": "new comment",
      "evidence_id": 3662,
      "id": 4702,
      "source_id": 3,
      "source_page": null,
      "confidence": "contested"
    },
    {
      "comment": "new comment 2",
      "evidence_id": 3662,
      "id": 4703,
      "source_id": 1,
      "source_page": "test",
      "confidence": "probable"
    }
  ]
\end{verbatim}}
\pagebreak[2]{\large\noindent\verb!/rest/evidence/<int:evidence_id>/tags!\hfill{}\tikz\node[inner sep=2pt,rounded corners=2pt,draw]{\small\texttt{GET}}; \tikz\node[inner sep=2pt,rounded corners=2pt,draw]{\small\texttt{HEAD}}; \tikz\node[inner sep=2pt,rounded corners=2pt,draw]{\small\texttt{OPTIONS}}; \tikz\node[inner sep=2pt,rounded corners=2pt,draw]{\small\texttt{PUT}};}
\nopagebreak[4]
{\small\begin{verbatim}Get or set tag set for evidence.

@param              evidence_id     ID of evidence

GET

    Get list of tag IDs as array.

    @returns            application/json

    Return value example for `GET /rest/evidence/1/tags`:

      [1,4,6]


PUT

    Replace list of tag IDs. Takes a JSON array as payload.

    @returns            nothing

    Payload value example for `PUT /rest/evidence/1/tags`:

      [1,2]
\end{verbatim}}
\pagebreak[2]{\large\noindent\verb!/rest/find-alternative-names!\hfill{}\tikz\node[inner sep=2pt,rounded corners=2pt,draw]{\small\texttt{OPTIONS}}; \tikz\node[inner sep=2pt,rounded corners=2pt,draw]{\small\texttt{POST}};}
\nopagebreak[4]
{\small\begin{verbatim}Retrieve `place.id` where the alternative name matches the `regex`.

@payload                application/json
  @param `regex`        ECMA-Script regular expression
  @param `ignore_case`  If `true`, regex is case-insensitive

@returns                application/json

As alternative names are not loaded initially, this API endpoint is used to
retrieve places' `id` for text search in the location list, where an
alternative name matches the search term `regex`. This API endpoint then
returns a JSON array of `place.id` for matching places.

Example response for case-insensitive serarch of "ko[a-g]":

  > POST /rest/find-alternative-names HTTP/1.1
  > Content-Type: application/json
  >
  > {
  >   "ignore_case": true,
  >   "regex": "ko[a-g]"
  > }
  >
  ---
  < HTTP/1.1 200 OK
  < Content-Type: application/json
  <
  < [
  <   236,
  <   694,
  <   493
  < ]
\end{verbatim}}
\pagebreak[2]{\large\noindent\verb!/rest/languages-list!\hfill{}\tikz\node[inner sep=2pt,rounded corners=2pt,draw]{\small\texttt{GET}}; \tikz\node[inner sep=2pt,rounded corners=2pt,draw]{\small\texttt{HEAD}}; \tikz\node[inner sep=2pt,rounded corners=2pt,draw]{\small\texttt{OPTIONS}};}
\nopagebreak[4]
{\small\begin{verbatim}Get a list of all religions.

@returns        application/json

This returns a JSON array of objects with `id` and `name` from table
`language`. This API endpoint replaces `/LanguagesList` in the old servlet
implementation.

Example return value excerpt:

  [
    {
      "id": 1,
      "name": "Arabic - DMG"
    },
    {
      "id": 2,
      "name": "Arabic - AS"
    },
    ...
  ]

\end{verbatim}}
\pagebreak[2]{\large\noindent\verb!/rest/person-instance/<int:person_instance_id>!\hfill{}\tikz\node[inner sep=2pt,rounded corners=2pt,draw]{\small\texttt{DELETE}}; \tikz\node[inner sep=2pt,rounded corners=2pt,draw]{\small\texttt{GET}}; \tikz\node[inner sep=2pt,rounded corners=2pt,draw]{\small\texttt{HEAD}}; \tikz\node[inner sep=2pt,rounded corners=2pt,draw]{\small\texttt{OPTIONS}}; \tikz\node[inner sep=2pt,rounded corners=2pt,draw]{\small\texttt{PATCH}}; \tikz\node[inner sep=2pt,rounded corners=2pt,draw]{\small\texttt{PUT}};}
\nopagebreak[4]
{\small\begin{verbatim}CRUD endpoint to manipulate person instances.

[all]     @param person_instance_id      ID of person instance, 0 or `None` for PUT

C/PUT     @payload                application/json
          @returns                application/json

Create a new person instance. `person_id` is a required field.
`confidence`, `comment`, and `annotation_id` are optional. Returns the ID
for the created person instance.

Exemplary payload for `PUT /person-instance/0`:

  {
    "person_id": 12,
    "confidence": "certain",
    "comment": "foo bar"
  }


R/GET     @returns                application/json

Get one person instance.

Exemplary reply for `GET /person-instance/64`:

  {
      "id": 64,
      "confidence": "certain",
      "comment": "baz",
      "annotation_id": null
  }


U/PATCH   @payload                application/json
          @returns                application/json

Update one or more of the fields `person_id`, `comment`, `confidence`, or
`annotation_id`.

Exemplary payload for `PATCH /person-instance/12345`:

  {
    "comment": "updated comment...",
  }


D/DELETE  @returns                application/json

Delete person instance. Write `user_action` log, return a JSON with all
deleted IDs.
\end{verbatim}}
\pagebreak[2]{\large\noindent\verb!/rest/person-list!\hfill{}\tikz\node[inner sep=2pt,rounded corners=2pt,draw]{\small\texttt{GET}}; \tikz\node[inner sep=2pt,rounded corners=2pt,draw]{\small\texttt{HEAD}}; \tikz\node[inner sep=2pt,rounded corners=2pt,draw]{\small\texttt{OPTIONS}};}
\nopagebreak[4]
{\small\begin{verbatim}Get content of table `person` as a list of dicts.

@return     application/json

Example return value excerpt:

    [
      {
        "comment": null,
        "id": 23,
        "name": "John bar Hebraye of Tarsus (appr. 667)",
        "person_type": 2,
        "time_range": ""
      },
      ...
    ]
\end{verbatim}}
\pagebreak[2]{\large\noindent\verb!/rest/person-type-list!\hfill{}\tikz\node[inner sep=2pt,rounded corners=2pt,draw]{\small\texttt{GET}}; \tikz\node[inner sep=2pt,rounded corners=2pt,draw]{\small\texttt{HEAD}}; \tikz\node[inner sep=2pt,rounded corners=2pt,draw]{\small\texttt{OPTIONS}};}
\nopagebreak[4]
{\small\begin{verbatim}Get a list of all person types.

@returns        application/json

This returns a JSON array of objects with `id`, and `type` from table
`person_type`.

Example return value excerpt:

  [
    {
      "id": 1,
      "type": "Bishop"
    },
    ...
  ]

\end{verbatim}}
\pagebreak[2]{\large\noindent\verb!/rest/person/<int:person_id>!\hfill{}\tikz\node[inner sep=2pt,rounded corners=2pt,draw]{\small\texttt{DELETE}}; \tikz\node[inner sep=2pt,rounded corners=2pt,draw]{\small\texttt{GET}}; \tikz\node[inner sep=2pt,rounded corners=2pt,draw]{\small\texttt{HEAD}}; \tikz\node[inner sep=2pt,rounded corners=2pt,draw]{\small\texttt{OPTIONS}}; \tikz\node[inner sep=2pt,rounded corners=2pt,draw]{\small\texttt{PATCH}}; \tikz\node[inner sep=2pt,rounded corners=2pt,draw]{\small\texttt{PUT}};}
\nopagebreak[4]
{\small\begin{verbatim}CRUD endpoint to manipulate person tuples.

[all]     @param person_id        ID of person tuple, 0 or `None` for PUT

C/PUT     @payload                application/json
          @returns                application/json

Create a new person tuple. `name` is a required field, the rest is optional.
Returns the ID for the created entity. Fails if a person with that name
already exists.

Exemplary payload for `PUT /person/0`:

  {
    "name": "Testperson",
    "comment": "Test comment",
    "time_range": "6th century",
    "person_type": 2
  }


R/GET     @returns                application/json

Get person data for the person with ID `person_id`.

@param person_id     Integer, `id` in table `person`
@returns            application/json

This returns the data from table `person` as a single JSON object.

Example return value:

  {
    "id": 12,
    "name": "Testperson",
    "comment": "Test comment",
    "time_range": "6th century",
    "person_type": 2
  }


U/PATCH   @payload                application/json
          @returns                application/json

Update one or more of the fields 'comment', 'name', 'time_range',
'person_type', or 'name'.

Exemplary payload for `PATCH /person/12345`:

  {
    "comment": "updated comment...",
    "name": "updated name"
  }


D/DELETE  @returns                application/json

Delete a person if there are no conflicts. Otherwise, fail. Returns the ID
of the deleted tuple.

\end{verbatim}}
\pagebreak[2]{\large\noindent\verb!/rest/place-instance/<int:place_instance_id>!\hfill{}\tikz\node[inner sep=2pt,rounded corners=2pt,draw]{\small\texttt{DELETE}}; \tikz\node[inner sep=2pt,rounded corners=2pt,draw]{\small\texttt{GET}}; \tikz\node[inner sep=2pt,rounded corners=2pt,draw]{\small\texttt{HEAD}}; \tikz\node[inner sep=2pt,rounded corners=2pt,draw]{\small\texttt{OPTIONS}}; \tikz\node[inner sep=2pt,rounded corners=2pt,draw]{\small\texttt{PATCH}}; \tikz\node[inner sep=2pt,rounded corners=2pt,draw]{\small\texttt{PUT}};}
\nopagebreak[4]
{\small\begin{verbatim}CRUD endpoint to manipulate place instances.

[all]     @param place_instance_id      ID of place instance, 0 or `None` for PUT

C/PUT     @payload                application/json
          @returns                application/json

Create a new place instance. `place_id` is a required field.
`confidence`, `comment`, and `annotation_id` are optional. Returns the ID
for the created place instance.

Exemplary payload for `PUT /place-instance/0`:

  {
    "place_id": 12,
    "confidence": "certain",
    "comment": "foo bar"
  }


R/GET     @returns                application/json

Get one place instance.

Exemplary reply for `GET /place-instance/64`:

  {
      "id": 64,
      "confidence": "certain",
      "comment": "baz",
      "annotation_id": null
  }


U/PATCH   @payload                application/json
          @returns                application/json

Update one or more of the fields `place_id`, `comment`, `confidence`, or
`annotation_id`.

Exemplary payload for `PATCH /place-instance/12345`:

  {
    "comment": "updated comment...",
  }


D/DELETE  @returns                application/json

Delete place instance. Write `user_action` log, return a JSON with all
deleted IDs.
\end{verbatim}}
\pagebreak[2]{\large\noindent\verb!/rest/place-list!\hfill{}\tikz\node[inner sep=2pt,rounded corners=2pt,draw]{\small\texttt{GET}}; \tikz\node[inner sep=2pt,rounded corners=2pt,draw]{\small\texttt{HEAD}}; \tikz\node[inner sep=2pt,rounded corners=2pt,draw]{\small\texttt{OPTIONS}};}
\nopagebreak[4]
{\small\begin{verbatim}Get a list of places.

@arg        filter  An array of arrays specifying an advanced filter
@returns            application/json

This returns all tuples from the view `place_overview` as a JSON array of
objects. The overview contains geographical location, place ID, location
confidence, place name, and place type name.
If `filter` is specified, the resulting values are restricted.

Example return value excerpt:

    [
      {
        "geoloc": {
          "lat": 36.335,
          "lng": 43.11889
        },
        "id": 1,
        "location_confidence": null,
        "name": "Mosul",
        "place_type": "Settlement"
      },
      ...
    ]
\end{verbatim}}
\pagebreak[2]{\large\noindent\verb!/rest/place-list-detailed!\hfill{}\tikz\node[inner sep=2pt,rounded corners=2pt,draw]{\small\texttt{GET}}; \tikz\node[inner sep=2pt,rounded corners=2pt,draw]{\small\texttt{HEAD}}; \tikz\node[inner sep=2pt,rounded corners=2pt,draw]{\small\texttt{OPTIONS}};}
\nopagebreak[4]
{\small\begin{verbatim}Get a list of places, with more details.

@returns            application/json

Example return value excerpt:

    [
      {
        "external_uris": [
          "IndAnat:37356",
          "https://nisanyanmap.com/?yer=37356",
          "syriaca:285",
          "https://syriaca.org/place/285",
          "EI2:SIM_0749",
          "http://dx.doi.org/10.1163/1573-3912_islam_SIM_0749",
          "EI1:SIM_0872",
          "http://dx.doi.org/10.1163/2214-871X_ei1_SIM_0872",
          "EI3:COM_23768",
          "http://dx.doi.org/10.1163/1573-3912_ei3_COM_23768"
        ],
        "name_vars": [
          "Arzun, Arzon",
          "Arzūn, Arzōn",
          "...",
          "Arzan",
          "..."
        ],
        "place_comment": "There are two Arzan. [...]",
        "place_id": 36,
        "place_name": "Arzan"
      },
      ...
    ]
\end{verbatim}}
\pagebreak[2]{\large\noindent\verb!/rest/place-set!\hfill{}\tikz\node[inner sep=2pt,rounded corners=2pt,draw]{\small\texttt{GET}}; \tikz\node[inner sep=2pt,rounded corners=2pt,draw]{\small\texttt{HEAD}}; \tikz\node[inner sep=2pt,rounded corners=2pt,draw]{\small\texttt{OPTIONS}}; \tikz\node[inner sep=2pt,rounded corners=2pt,draw]{\small\texttt{POST}};}
\nopagebreak[4]
{\small\begin{verbatim}REST endpoint for place sets. Place sets are used as a filtering
possibility in the visualization, and are stored in the database to be
shared between users.

GET         Returns a JSON list of all place sets in the database.
POST        Accepts ONE JSON place set as a payload. Depending on whether
            the UUID exists in the database already, the entry is either
            overwritten, or a new entry is created.
\end{verbatim}}
\pagebreak[2]{\large\noindent\verb!/rest/place-type-list!\hfill{}\tikz\node[inner sep=2pt,rounded corners=2pt,draw]{\small\texttt{GET}}; \tikz\node[inner sep=2pt,rounded corners=2pt,draw]{\small\texttt{HEAD}}; \tikz\node[inner sep=2pt,rounded corners=2pt,draw]{\small\texttt{OPTIONS}};}
\nopagebreak[4]
{\small\begin{verbatim}Get a list of all place types.

@returns        application/json

This returns a JSON array of objects with `id`, `type`, and `visible` from
table `place_type`. This API endpoint replaces `/PlaceTypeList` in the old
servlet implementation.

Example return value excerpt:

  [
    {
      "id": 1,
      "type": "Unknown",
      "visible": true
    },
    ...
  ]

\end{verbatim}}
\pagebreak[2]{\large\noindent\verb!/rest/place/<int:place_id>!\hfill{}\tikz\node[inner sep=2pt,rounded corners=2pt,draw]{\small\texttt{DELETE}}; \tikz\node[inner sep=2pt,rounded corners=2pt,draw]{\small\texttt{GET}}; \tikz\node[inner sep=2pt,rounded corners=2pt,draw]{\small\texttt{HEAD}}; \tikz\node[inner sep=2pt,rounded corners=2pt,draw]{\small\texttt{OPTIONS}}; \tikz\node[inner sep=2pt,rounded corners=2pt,draw]{\small\texttt{PATCH}}; \tikz\node[inner sep=2pt,rounded corners=2pt,draw]{\small\texttt{PUT}};}
\nopagebreak[4]
{\small\begin{verbatim}CRUD endpoint to manipulate place tuples.

[all]     @param place_id         ID of place tuple, 0 or `None` for PUT

C/PUT     @payload                application/json
          @returns                application/json

Create a new place tuple. `name` is a required field, the rest is optional.
Returns the ID for the created entity. Fails if a place with that name
already exists.

Exemplary payload for `PUT /place/0`:

  {
    "name": "Testplace",
    "comment": "Test comment",
    "geoloc": "(48.2,9.6)",
    "confidence": "contested",
    "visible": true,
    "place_type_id": 2
  }


R/GET     @returns                application/json

Get place data for the place with ID `place_id`.

@param place_id     Integer, `id` in table `place`
@returns            application/json

This returns the data from table `place` as a single JSON object.

Example return value (2020-01-09) of `GET /place/12`:

    {
        "comment": "Gesch\u00e4tzt nach Iraq and the Persian Gulf [...]",
        "confidence": null,
        "geoloc": "(33.542,44.3726)",
        "geonames": "...",
        "google": "...",
        "id": 12,
        "name": "al-Baradan",
        "place_type_id": 3,
        "syriaca": null,
        "visible": true
    }



U/PATCH   @payload                application/json
          @returns                application/json

Update one or more of the fields 'comment', 'confidence', 'visible',
'place_type_id', 'geoloc', or 'name'.

Exemplary payload for `PATCH /place/12345`:

  {
    "visible": false,
    "comment": "updated comment..."
  }


D/DELETE  @returns                application/json

Delete a place if there are no conflicts. Otherwise, fail. Returns the ID
of the deleted tuple.

\end{verbatim}}
\pagebreak[2]{\large\noindent\verb!/rest/place/<int:place_id>/alternative-name/<int:name_id>!\hfill{}\tikz\node[inner sep=2pt,rounded corners=2pt,draw]{\small\texttt{DELETE}}; \tikz\node[inner sep=2pt,rounded corners=2pt,draw]{\small\texttt{GET}}; \tikz\node[inner sep=2pt,rounded corners=2pt,draw]{\small\texttt{HEAD}}; \tikz\node[inner sep=2pt,rounded corners=2pt,draw]{\small\texttt{OPTIONS}}; \tikz\node[inner sep=2pt,rounded corners=2pt,draw]{\small\texttt{PATCH}}; \tikz\node[inner sep=2pt,rounded corners=2pt,draw]{\small\texttt{PUT}};}
\nopagebreak[4]
{\small\begin{verbatim}CRUD endpoint to manipulate alternative names.

[all]     @param place_id         ID of place tuple
[all]     @param name_id          ID of name_var tuple, 0 or `None` for PUT

C/PUT     @payload                application/json
          @returns                application/json

Create new alternative name. `name` and `language_id` are required

Exemplary payload for `PUT /place/1234/alternative-name/0`:

  {
    "name": "Testplace",
    "language_id": 2
  }


R/GET     @returns                application/json

Get alternative name tuple.

Example return value of `GET /place/1234/alternative-name/5678`:

    {
      "id": 5678,
      "place_id": 1234,
      "name": "Dummy test name",
      "language_id": 12
    }


U/PATCH   @payload                application/json
          @returns                application/json

Update one or more of the fields 'name' and 'language_id'.

Exemplary payload for `PATCH /place/alternative-name/5678`:

  {
    "name": "New name",
    "language_id": 13
  }


D/DELETE  @returns                application/json

Delete an alternative name tuple. Returns the ID of the deleted tuple.
\end{verbatim}}
\pagebreak[2]{\large\noindent\verb!/rest/place/<int:place_id>/alternative-name/all!\hfill{}\tikz\node[inner sep=2pt,rounded corners=2pt,draw]{\small\texttt{GET}}; \tikz\node[inner sep=2pt,rounded corners=2pt,draw]{\small\texttt{HEAD}}; \tikz\node[inner sep=2pt,rounded corners=2pt,draw]{\small\texttt{OPTIONS}};}
\nopagebreak[4]
{\small\begin{verbatim}Get a list of all alternative names for place with ID `place_id`.

@returns                          application/json

Exemplary return value for `GET /place/1234/alternative-name/all`:

  [
    {
      "id": 5678,
      "place_id": 1234,
      "name": "Dummy test name",
      "language_id": 12
    },
    {
      "id": 5679,
      "place_id": 1234,
      "name": "Dummy test name 2",
      "language_id": 4
    },
    ...
  ]
\end{verbatim}}
\pagebreak[2]{\large\noindent\verb!/rest/place/<int:place_id>/details!\hfill{}\tikz\node[inner sep=2pt,rounded corners=2pt,draw]{\small\texttt{GET}}; \tikz\node[inner sep=2pt,rounded corners=2pt,draw]{\small\texttt{HEAD}}; \tikz\node[inner sep=2pt,rounded corners=2pt,draw]{\small\texttt{OPTIONS}};}
\nopagebreak[4]
{\small\begin{verbatim}Retrieve more details for the place with ID `place_id`.

@param `place_id`   `id` in table `place`
@returns            application/json

This API endpoint is aimed at the location list of the visualization, where
tooltips show more details for a place on hover. This information is not
loaded from the server initially for efficiency reasons. Instead, it is
queried when the tooltip is created.
As of now (2020-01-10), the call returns a JSON object containing the place
ID, the `comment` field from table `place`, and an array of alternative
names for the place together with the respective language name. This will
exclude alternative names that are not main forms.

Example return value excerpt for `GET /place/14/details`:

  {
    "alternative_names": [
      {
        "language": "Arabic - AS",
        "name": "\u062d\u0644\u0628"
      },
      {
        "language": "Arabic - DMG",
        "name": "\u1e24alab"
      },
      {
        "language": "Syriac - SS",
        "name": "\u071a\u0720\u0712 "
      },
      ...
    ],
    "comment": "[dummy] Comments are strings or null",
    "place_id": 14,
    "external_uris": [
      "Syriaca:3055",
      ...
    ],
  }
\end{verbatim}}
\pagebreak[2]{\large\noindent\verb!/rest/place/<int:place_id>/evidence!\hfill{}\tikz\node[inner sep=2pt,rounded corners=2pt,draw]{\small\texttt{GET}}; \tikz\node[inner sep=2pt,rounded corners=2pt,draw]{\small\texttt{HEAD}}; \tikz\node[inner sep=2pt,rounded corners=2pt,draw]{\small\texttt{OPTIONS}};}
\nopagebreak[4]
{\small\begin{verbatim}Retrieve religion evidence for the place with ID `place_id`.

@param `place_id`   `id` in table `place`
@returns            application/json

This replaces the `/Religions` endpoint in the old servlet implementation.

Returns a list of evidences, with comments and confidences, time spans, and
sources.

Example return value excerpt for `GET /place/12/evidence`:

  {
    "place_id": 12,
    "evidence": [
      {
        "evidence_id": 17,
        "interpretation_confidence": null,
        "place_attribution_confidence": null,
        "place_id": 12,
        "place_instance_comment": null,
        "religion_confidence": null,
        "religion_id": 5,
        "religion_instance_comment": null,
        "religion_name": "Church of the East",
        "sources": [
          {
            "source_id": 8,
            "source_instance_comment": "Neuer Eintrag",
            "source_name": "AKg = Jedin, Hubert; Martin, Jochen (Hg.) (1987): [...]",
            "source_page": "26"
          }
        ],
        "time_spans": [
          {
            "comment": null,
            "confidence": null,
            "end": 1200,
            "start": 800
          }
        ]
      },
      ...
    ]
  }
\end{verbatim}}
\pagebreak[2]{\large\noindent\verb!/rest/place/<int:place_id>/evidence-ids!\hfill{}\tikz\node[inner sep=2pt,rounded corners=2pt,draw]{\small\texttt{GET}}; \tikz\node[inner sep=2pt,rounded corners=2pt,draw]{\small\texttt{HEAD}}; \tikz\node[inner sep=2pt,rounded corners=2pt,draw]{\small\texttt{OPTIONS}};}
\nopagebreak[4]
{\small\begin{verbatim}Get all evidence tuple IDs for the place with ID `place_id`.

@returns application/json

Example return value exerpt for `/place/12/evidence-ids`:

  [
    1,
    5,
    12,
    191,
    ...
  ]
\end{verbatim}}
\pagebreak[2]{\large\noindent\verb!/rest/place/<int:place_id>/external-uri-list!\hfill{}\tikz\node[inner sep=2pt,rounded corners=2pt,draw]{\small\texttt{GET}}; \tikz\node[inner sep=2pt,rounded corners=2pt,draw]{\small\texttt{HEAD}}; \tikz\node[inner sep=2pt,rounded corners=2pt,draw]{\small\texttt{OPTIONS}};}
\nopagebreak[4]
{\small\begin{verbatim}Get a list of external URIs for a place.

@param   place_id   ID of place
@returns            application/json

Example return value excerpt for `GET /rest/place/12/external-uri-list`:

    [
      {
        "id": 1,
        "place_id": 12,
        "uri_namespace_id": 1,
        "uri_fragment": "27223",
        "comment": null
      }
      ...
    ]
\end{verbatim}}
\pagebreak[2]{\large\noindent\verb!/rest/place/<int:place_id>/external-uri-list!\hfill{}\tikz\node[inner sep=2pt,rounded corners=2pt,draw]{\small\texttt{GET}}; \tikz\node[inner sep=2pt,rounded corners=2pt,draw]{\small\texttt{HEAD}}; \tikz\node[inner sep=2pt,rounded corners=2pt,draw]{\small\texttt{OPTIONS}};}
\nopagebreak[4]
{\small\begin{verbatim}Get a list of external URIs for a place.

@param   place_id   ID of place
@returns            application/json

Example return value excerpt for `GET /rest/place/12/external-uri-list`:

    [
      {
        "id": 1,
        "place_id": 12,
        "uri_namespace_id": 1,
        "uri_fragment": "27223",
        "comment": null
      }
      ...
    ]
\end{verbatim}}
\pagebreak[2]{\large\noindent\verb!/rest/place/all!\hfill{}\tikz\node[inner sep=2pt,rounded corners=2pt,draw]{\small\texttt{GET}}; \tikz\node[inner sep=2pt,rounded corners=2pt,draw]{\small\texttt{HEAD}}; \tikz\node[inner sep=2pt,rounded corners=2pt,draw]{\small\texttt{OPTIONS}};}
\nopagebreak[4]
{\small\begin{verbatim}Get content of table `place` as a list of dicts.

@return     application/json

Example return value excerpt:

    [
      {
        "comment": "Unknown place",
        "confidence": null,
        "geoloc": null,
        "id": 448,
        "name": "Dirigh",
        "place_type_id": 4,
        "visible": true
      },
      ...
    ]
\end{verbatim}}
\pagebreak[2]{\large\noindent\verb!/rest/places!\hfill{}\tikz\node[inner sep=2pt,rounded corners=2pt,draw]{\small\texttt{GET}}; \tikz\node[inner sep=2pt,rounded corners=2pt,draw]{\small\texttt{HEAD}}; \tikz\node[inner sep=2pt,rounded corners=2pt,draw]{\small\texttt{OPTIONS}};}
\nopagebreak[4]
{\small\begin{verbatim}Get a list of places and their IDs.

@returns            application/json

This returns a list of place names and IDs.

Example return value excerpt:

    [
      {
        "id": 1,
        "name": "Mosul",
      },
      ...
    ]
\end{verbatim}}
\pagebreak[2]{\large\noindent\verb!/rest/religion-instance/<int:religion_instance_id>!\hfill{}\tikz\node[inner sep=2pt,rounded corners=2pt,draw]{\small\texttt{DELETE}}; \tikz\node[inner sep=2pt,rounded corners=2pt,draw]{\small\texttt{GET}}; \tikz\node[inner sep=2pt,rounded corners=2pt,draw]{\small\texttt{HEAD}}; \tikz\node[inner sep=2pt,rounded corners=2pt,draw]{\small\texttt{OPTIONS}}; \tikz\node[inner sep=2pt,rounded corners=2pt,draw]{\small\texttt{PATCH}}; \tikz\node[inner sep=2pt,rounded corners=2pt,draw]{\small\texttt{PUT}};}
\nopagebreak[4]
{\small\begin{verbatim}CRUD endpoint to manipulate religion instances.

[all]     @param religion_instance_id      ID of religion instance, 0 or `None` for PUT

C/PUT     @payload                application/json
          @returns                application/json

Create a new religion instance. `religion_id` is a required field.
`confidence`, `comment`, and `annotation_id` are optional. Returns the ID
for the created religion instance.

Exemplary payload for `PUT /religion-instance/0`:

  {
    "religion_id": 12,
    "confidence": "certain",
    "comment": "foo bar"
  }


R/GET     @returns                application/json

Get one religion instance.

Exemplary reply for `GET /religion-instance/64`:

  {
      "id": 64,
      "confidence": "certain",
      "comment": "baz",
      "annotation_id": null
  }


U/PATCH   @payload                application/json
          @returns                application/json

Update one or more of the fields `religion_id`, `comment`, `confidence`, or
`annotation_id`.

Exemplary payload for `PATCH /religion-instance/12345`:

  {
    "comment": "updated comment...",
  }


D/DELETE  @returns                application/json

Delete religion instance. Write `user_action` log, return a JSON with all
deleted IDs.
\end{verbatim}}
\pagebreak[2]{\large\noindent\verb!/rest/religion-list!\hfill{}\tikz\node[inner sep=2pt,rounded corners=2pt,draw]{\small\texttt{GET}}; \tikz\node[inner sep=2pt,rounded corners=2pt,draw]{\small\texttt{HEAD}}; \tikz\node[inner sep=2pt,rounded corners=2pt,draw]{\small\texttt{OPTIONS}};}
\nopagebreak[4]
{\small\begin{verbatim}Get a list of religion names and IDs.

@returns            application/json

Example return value excerpt:

    [
      {
        "id": 1,
        "name": "Christianity",
        "parent_id": null
      },
      {
        "id": 5,
        "name": "Church of the East",
        "parent_id": 1
      },
      ...
    ]
\end{verbatim}}
\pagebreak[2]{\large\noindent\verb!/rest/religions!\hfill{}\tikz\node[inner sep=2pt,rounded corners=2pt,draw]{\small\texttt{GET}}; \tikz\node[inner sep=2pt,rounded corners=2pt,draw]{\small\texttt{HEAD}}; \tikz\node[inner sep=2pt,rounded corners=2pt,draw]{\small\texttt{OPTIONS}};}
\nopagebreak[4]
{\small\begin{verbatim}Get a hierarchy of religions.

@returns            application/json

This utilizes the `parent_id` attribute in the table `religion` to build a
list of trees. The return value is an array of JSON objects. Each root node
is a main religion, and one of its attributes is `children`, holding an
array of children, which may again hold children, and so on.

Each node of each tree has the following, exemplary (2020-01-09) structure:

    {
        "abbreviation": "MARO",
        "children": [ ... ],
        "color": "hsl(10, 80%, 50%)",
        "id": 3,
        "name": "Maronite Church",
        "parent_id": 98
    }
\end{verbatim}}
\pagebreak[2]{\large\noindent\verb!/rest/source-instance/<int:source_instance_id>!\hfill{}\tikz\node[inner sep=2pt,rounded corners=2pt,draw]{\small\texttt{DELETE}}; \tikz\node[inner sep=2pt,rounded corners=2pt,draw]{\small\texttt{GET}}; \tikz\node[inner sep=2pt,rounded corners=2pt,draw]{\small\texttt{HEAD}}; \tikz\node[inner sep=2pt,rounded corners=2pt,draw]{\small\texttt{OPTIONS}}; \tikz\node[inner sep=2pt,rounded corners=2pt,draw]{\small\texttt{PATCH}}; \tikz\node[inner sep=2pt,rounded corners=2pt,draw]{\small\texttt{PUT}};}
\nopagebreak[4]
{\small\begin{verbatim}CRUD endpoint to manipulate source instance entries.

[all]     @param source_instance_id     ID of tuple, ignored for PUT (use 0)

C/PUT     @payload                      application/json
          @returns                      application/json

Create a new source instance tuple. `source_id` is required in the payload,
`source_page` and `comment` are optional. Returns the `id` of the created
tuple on success.

Exemplary payload for `PUT /source-instance/0`:

  {
    "comment": "new instance via PUT, 2",
    "source_id": 3,
    "source_page": "1--2",
    "source_confidence": "contested",
    "evidence_id": 12
  }



R/GET     @returns                      application/json

Get one source instance tuple.

Exemplary return value for `GET /source-instance/4702`:

  {
    "comment": "new comment",
    "evidence_id": 3662,
    "id": 4702,
    "source_id": 3,
    "source_page": null,
    "source_confidence": "contested"
  }


U/PATCH   @payload                application/json
          @returns                application/json

Update one or more of the fields `comment` or `source_page`. All other fields or
connected entities must be modified via their respective endpoints.

Exemplary payload for `PATCH /source-instance/567`:

  {
    "comment": "updated comment...",
    "source_page": "6--7"
  }

D/DELETE  @returns                application/json

Delete all related entities, write `user_action` log, return a JSON with all deleted IDs.

Exemplary return valye for `DELETE /source-instance/5678`:

  {
    "deleted": {
      "source_instance": 5678
    }
  }

\end{verbatim}}
\pagebreak[2]{\large\noindent\verb!/rest/sources-list!\hfill{}\tikz\node[inner sep=2pt,rounded corners=2pt,draw]{\small\texttt{GET}}; \tikz\node[inner sep=2pt,rounded corners=2pt,draw]{\small\texttt{HEAD}}; \tikz\node[inner sep=2pt,rounded corners=2pt,draw]{\small\texttt{OPTIONS}};}
\nopagebreak[4]
{\small\begin{verbatim}Get a list of all sources.

@returns        application/json

This returns a JSON array of objects with `id` and `name` from table
`source`. This API endpoint replaces `/SourcesList` in the old servlet
implementation.

Example return value excerpt:

  [
    {
      "id": 32,
      "name": "Bar Ebroyo / Wilmshurst",
      "short": "Bar Ebroyo / Wilmshurst",
      "default_confidence": "probable"
    },
    {
      "id": 1,
      "name": "Fiey, Jean Maurice (1993): Pour un Oriens [...]",
      "short": "OCN",
      "default_confidence": "certain"
    },
    ...
  ]
\end{verbatim}}
\pagebreak[2]{\large\noindent\verb!/rest/tag-list!\hfill{}\tikz\node[inner sep=2pt,rounded corners=2pt,draw]{\small\texttt{GET}}; \tikz\node[inner sep=2pt,rounded corners=2pt,draw]{\small\texttt{HEAD}}; \tikz\node[inner sep=2pt,rounded corners=2pt,draw]{\small\texttt{OPTIONS}};}
\nopagebreak[4]
{\small\begin{verbatim}Get a list of tags.

@returns            application/json

Example return value excerpt:

  [
    {
      "id": 1,
      "tagname": "bishopric",
      "comment": "test comment"
    },
    ...
  ]
\end{verbatim}}
\pagebreak[2]{\large\noindent\verb!/rest/tag-sets!\hfill{}\tikz\node[inner sep=2pt,rounded corners=2pt,draw]{\small\texttt{GET}}; \tikz\node[inner sep=2pt,rounded corners=2pt,draw]{\small\texttt{HEAD}}; \tikz\node[inner sep=2pt,rounded corners=2pt,draw]{\small\texttt{OPTIONS}};}
\nopagebreak[4]
{\small\begin{verbatim}Get the set of evidence IDs for each tag.

@returns            application/json

Example return value excerpt:

  [
    {
      "tag_id": 1,
      "evidence_ids": [1,2,3,6,37]
    },
    {
      "tag_id": 2,
      "evidence_ids": [1,3]
    },
    ...
  ]
\end{verbatim}}
\pagebreak[2]{\large\noindent\verb!/rest/time-group/<int:time_group_id>!\hfill{}\tikz\node[inner sep=2pt,rounded corners=2pt,draw]{\small\texttt{DELETE}}; \tikz\node[inner sep=2pt,rounded corners=2pt,draw]{\small\texttt{GET}}; \tikz\node[inner sep=2pt,rounded corners=2pt,draw]{\small\texttt{HEAD}}; \tikz\node[inner sep=2pt,rounded corners=2pt,draw]{\small\texttt{OPTIONS}}; \tikz\node[inner sep=2pt,rounded corners=2pt,draw]{\small\texttt{PATCH}}; \tikz\node[inner sep=2pt,rounded corners=2pt,draw]{\small\texttt{PUT}};}
\nopagebreak[4]
{\small\begin{verbatim}CRUD endpoint to manipulate time_group entries.

[all]     @param time_span_id           ID of tuple, ignored for PUT (use 0)

C/PUT     @payload                      application/json
          @returns                      application/json

Create a new group tuple. Optionally takes a valid `annotation_id`.
Returns the `id` of the created tuple on success.

Exemplary payload for `PUT /time-group/0`:

  {
    "annotation_id": 412
  }



R/GET     @returns                      application/json

Get one time_group tuple.

Exemplary return value for `GET /time-group/3658`:

  {
    "annotation_id": null,
    "time_group_id": 3658,
    "time_spans": [
      {
        "comment": null,
        "confidence": null,
        "end": 988,
        "start": 985,
        "time_instance_id": 3658
      }
    ]
  }



U/PATCH   @payload                application/json
          @returns                204 NO CONTENT

Update the `annotation_id` field to a valid value, or `null`.

Exemplary payload for `PATCH /time-group/789`:

  {
    "annotation_id": 456
  }


D/DELETE  @returns                application/json

Delete all related entities, write `user_action` log, return a JSON with all deleted IDs.

Exemplary return value for `DELETE /time-group/5678`:

  {
    "deleted": {
      "annotation": 4211,
      "time_group": 5678,
      "time_instance": [
        19221,
        19222,
        32115
      ]
  }

\end{verbatim}}
\pagebreak[2]{\large\noindent\verb!/rest/time-group/<int:time_group_id>/time-instance/<int:time_instance_id>!\hfill{}\tikz\node[inner sep=2pt,rounded corners=2pt,draw]{\small\texttt{DELETE}}; \tikz\node[inner sep=2pt,rounded corners=2pt,draw]{\small\texttt{GET}}; \tikz\node[inner sep=2pt,rounded corners=2pt,draw]{\small\texttt{HEAD}}; \tikz\node[inner sep=2pt,rounded corners=2pt,draw]{\small\texttt{OPTIONS}}; \tikz\node[inner sep=2pt,rounded corners=2pt,draw]{\small\texttt{PATCH}}; \tikz\node[inner sep=2pt,rounded corners=2pt,draw]{\small\texttt{PUT}};}
\nopagebreak[4]
{\small\begin{verbatim}CRUD endpoint to manipulate time_instance entries. The `time_group_id` and
`time_instance.time_group_id` must match.

[all]     @param time_group_id          ID of time_group tuple
          @param time_instance_id       ID of time_instance tuple, ignored for PUT


C/PUT     @payload                      application/json
          @returns                      application/json

Create a new timespan tuple. Optionally takes start, end, comment and
confidence. Returns the ID of the created tuple.

Exemplary payload for `PUT /time-group/2/time-instance/0`:

  {
    "start": 1200,
    "end": 1300,
    "comment": "New comment",
    "confidence": "contested"
  }



R/GET     @returns                      application/json

Get one time_instance tuple.

Exemplary return value for `GET /time-group/3659/time-instance/3670`:

  {
    "comment": "new comment",
    "confidence": "probable",
    "end": 1299,
    "id": 3669,
    "start": null,
    "time_group_id": 3659
  }



U/PATCH   @payload                application/json
          @returns                204 NO CONTENT

Update the start, end, confidence, or comment fields.

Exemplary payload for `PATCH /time-group/789/time-instance/1000`:

  {
    "start": 1238,
    "comment": "Start year now confirmed",
    "confidence": "probable"
  }


D/DELETE  @returns                application/json

Delete tuple, write `user_action` log, return a JSON with deleted ID.

Exemplary return value for `DELETE /time-group/5678/time-instance/1234`:

  {
    "deleted": {
      "time_instance": 1234
    }
  }

\end{verbatim}}
\pagebreak[2]{\large\noindent\verb!/rest/uri/external-database-list!\hfill{}\tikz\node[inner sep=2pt,rounded corners=2pt,draw]{\small\texttt{GET}}; \tikz\node[inner sep=2pt,rounded corners=2pt,draw]{\small\texttt{HEAD}}; \tikz\node[inner sep=2pt,rounded corners=2pt,draw]{\small\texttt{OPTIONS}};}
\nopagebreak[4]
{\small\begin{verbatim}Get a list of external databases registered.

@returns            application/json

This returns a list of all external_database tuples in the database as an array.

Example return value excerpt:

    [
      {
        "id": 1,
        "name": "Foo: the foo database",
        "short_name": "Foo",
        "url": "http://foo.bar/",
        "comment": null
      },
      ...
    ]
\end{verbatim}}
\pagebreak[2]{\large\noindent\verb!/rest/uri/external-person-uri-list!\hfill{}\tikz\node[inner sep=2pt,rounded corners=2pt,draw]{\small\texttt{GET}}; \tikz\node[inner sep=2pt,rounded corners=2pt,draw]{\small\texttt{HEAD}}; \tikz\node[inner sep=2pt,rounded corners=2pt,draw]{\small\texttt{OPTIONS}};}
\nopagebreak[4]
{\small\begin{verbatim}Get a list of external person URIs.

@returns            application/json

Example return value excerpt for `GET /rest/uri/external-person-uri-list`:

    [
      {
        "id": 1,
        "person_id": 12,
        "uri_namespace_id": 1,
        "uri_fragment": "27223",
        "comment": null
      }
      ...
    ]
\end{verbatim}}
\pagebreak[2]{\large\noindent\verb!/rest/uri/external-person-uri/<int:uri_id>!\hfill{}\tikz\node[inner sep=2pt,rounded corners=2pt,draw]{\small\texttt{DELETE}}; \tikz\node[inner sep=2pt,rounded corners=2pt,draw]{\small\texttt{GET}}; \tikz\node[inner sep=2pt,rounded corners=2pt,draw]{\small\texttt{HEAD}}; \tikz\node[inner sep=2pt,rounded corners=2pt,draw]{\small\texttt{OPTIONS}}; \tikz\node[inner sep=2pt,rounded corners=2pt,draw]{\small\texttt{PATCH}}; \tikz\node[inner sep=2pt,rounded corners=2pt,draw]{\small\texttt{PUT}};}
\nopagebreak[4]
{\small\begin{verbatim}CRUD endpoint to manipulate external person URIs.

[all]     @param uri_id           ID of tuple, 0 or `None` for PUT

C/PUT     @payload                application/json
          @returns                application/json

Create a new person URI tuple.

Required fields: `person_id`, `uri_namespace_id`, `uri_fragment`.
Optional fields: `comment`.

Returns the ID for the created entity.

Exemplary payload for `PUT /uri/external-person-uri/0`:

  {
    "person_id": 12,
    "uri_namespace_id": 1,
    "uri_fragment": "1234",
    "comment": "comment"
  }


R/GET     @returns                application/json

Get data for the person URI.

@returns            application/json

Example return value of `GET /uri/external-person-uri/1`:

  {
    "id": 1,
    "person_id": 12,
    "uri_namespace_id": 1,
    "uri_fragment": "1234",
    "comment": "comment"
  }


U/PATCH   @payload                application/json
          @returns                205 RESET CONTENT

Update one or more of the fields `person_id`, `uri_namespace_id`,
`uri_fragment`, or `comment`.

Exemplary payload for `PATCH /uri/external-person-uri/12345`:

  {
    "uri_fragment": "1236",
    "comment": "updated comment..."
  }


D/DELETE  @returns                application/json

Delete the tuple and return the ID of the deleted tuple.
\end{verbatim}}
\pagebreak[2]{\large\noindent\verb!/rest/uri/external-place-uri/<int:uri_id>!\hfill{}\tikz\node[inner sep=2pt,rounded corners=2pt,draw]{\small\texttt{DELETE}}; \tikz\node[inner sep=2pt,rounded corners=2pt,draw]{\small\texttt{GET}}; \tikz\node[inner sep=2pt,rounded corners=2pt,draw]{\small\texttt{HEAD}}; \tikz\node[inner sep=2pt,rounded corners=2pt,draw]{\small\texttt{OPTIONS}}; \tikz\node[inner sep=2pt,rounded corners=2pt,draw]{\small\texttt{PATCH}}; \tikz\node[inner sep=2pt,rounded corners=2pt,draw]{\small\texttt{PUT}};}
\nopagebreak[4]
{\small\begin{verbatim}CRUD endpoint to manipulate external place URIs.

[all]     @param uri_id           ID of tuple, 0 or `None` for PUT

C/PUT     @payload                application/json
          @returns                application/json

Create a new place URI tuple.

Required fields: `place_id`, `uri_namespace_id`, `uri_fragment`.
Optional fields: `comment`.

Returns the ID for the created entity.

Exemplary payload for `PUT /uri/external-place-uri/0`:

  {
    "place_id": 12,
    "uri_namespace_id": 1,
    "uri_fragment": "1234",
    "comment": "comment"
  }


R/GET     @returns                application/json

Get data for the place URI.

@returns            application/json

Example return value of `GET /uri/external-place-uri/1`:

  {
    "id": 1,
    "place_id": 12,
    "uri_namespace_id": 1,
    "uri_fragment": "1234",
    "comment": "comment"
  }


U/PATCH   @payload                application/json
          @returns                205 RESET CONTENT

Update one or more of the fields `place_id`, `uri_namespace_id`,
`uri_fragment`, or `comment`.

Exemplary payload for `PATCH /uri/external-place-uri/12345`:

  {
    "uri_fragment": "1236",
    "comment": "updated comment..."
  }


D/DELETE  @returns                application/json

Delete the tuple and return the ID of the deleted tuple.
\end{verbatim}}
\pagebreak[2]{\large\noindent\verb!/rest/uri/uri-namespace-list!\hfill{}\tikz\node[inner sep=2pt,rounded corners=2pt,draw]{\small\texttt{GET}}; \tikz\node[inner sep=2pt,rounded corners=2pt,draw]{\small\texttt{HEAD}}; \tikz\node[inner sep=2pt,rounded corners=2pt,draw]{\small\texttt{OPTIONS}};}
\nopagebreak[4]
{\small\begin{verbatim}Get a list of URI namespaces registered.

@returns            application/json

This returns a list of all uri_namespace tuples in the database as an array.

Example return value excerpt:

    [
      {
        "id": 1,
        "external_database_id", 1,
        "uri_pattern": "https://first.db/place/%s",
        "short_name": "first:%s",
        "comment": null
      }
      ...
    ]
\end{verbatim}}
\pagebreak[2]
